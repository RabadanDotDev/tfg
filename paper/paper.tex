\documentclass[10pt,a4paper,twoside]{article}

\usepackage{graphicx}
\usepackage{amsmath}
\usepackage{multicol}
\usepackage{amssymb}
\usepackage{tabularx}
\usepackage{xcolor}
\usepackage{fontspec}
\usepackage{titlesec, blindtext}
\usepackage[spanish,es-tabla]{babel}
\usepackage{tocloft}
\usepackage[
    hyperindex=true,
    bookmarks=true,
    bookmarksnumbered=true,
    hidelinks,
]{hyperref}
\usepackage[all]{hypcap}
\usepackage[labelfont=bf]{caption}
\usepackage{float}
\usepackage{bytefield}
\usepackage{listings}
\usepackage[
    top=2cm,
    inner=2cm,
    outer=1.5cm,
    bottom=2.5cm,
]{geometry}
\usepackage[skip=5pt]{parskip}

\setmainfont[Ligature=TeX]{Times New Roman}
\setlength\columnsep{0.5cm}
\setlength{\parindent}{0in}
\makeatletter
\renewcommand{\fnum@figure}{Fig. \thefigure}
\makeatother
\captionsetup{labelsep=period}
\titleformat{\section}[block]{\centering\large\bfseries}{\thesection. }{0.1em}{}
\titlespacing{\section}{0em}{0.3em}{0.3em}
\pagenumbering{gobble}

\bibliographystyle{unsrt}

\begin{document}

\begin{center}
    \bfseries\fontsize{22pt}{27pt}\selectfont\par
    Desarrollo de una herramienta de análisis de tráfico de red y su uso en algoritmos de ML para la detección de ataques.
    \par
\end{center}

\bigskip

\begin{center}
    \large\par
    Raul Rabadan Arroyo
    \par
\end{center}

\begin{center}
    \par
    Estudiante en Grado de Ingeniería Informática de la EPSEVG-UPC
    \par
\end{center}

\smallskip

\begin{multicols}{2}
    \section*{Resumen}

    En este Trabajo de Final de Grado (TFG) se hace el desarrollo de una herramienta de análisis de red y se demuestra su funcionamiento a partir de analizar trazas de tráfico de red y el uso de los datos resultantes para entrenar modelos con Machine Learning. La herramienta desarrollada puede ser utilizada como componente en sistemas  de detección de intrusiones y tiene el potencial para ser extendida con más funcionalidades en el futuro. Se ha hecho uso de Rust para el desarrollo de la herramienta y es capaz de generar 72 características continuas, 2 discretas para el protocolo de transporte y una identificación de cada flujo compuesta por 7 valores. Adicionalmente, hay soporte para el etiquetado automático a partir de un fichero CSV. Con los registros generados por la herramienta, se ha podido realizar una detección de flujos de red malignos con una puntuación F1 media del 98.66\%.

    \section{Introducción}

    Por hacer.

    \section{CICFlowMeter}

    Por hacer. \cite{cicflowpost}

    \section{Desarollo herramienta}

    Por hacer.

    \section{Aplicación caso práctico ML}

    Por hacer.

    \section{Conclusiones}

    Por hacer.

    \section{Acerca de las figuras tablas}

    Las pautas para editar y situar las tablas son muy simples. Sólo hay que escribirlas como texto y situarlas como figuras. Se puede usar negrita y diferentes tamaños (desde Times New Roman 12 hasta Times New Roman 10) para diferenciar la información relevante que contienen, como está indicado debajo.
    
    Respecto a la anchura de las tablas, las pautas a seguir serán las mismas que en el caso de las figuras.

    \section{Conclusiones}

    Por hacer.

    \section{Agradecimientos}

    Por hacer.

    \bibliography{refs}

\end{multicols}
\end{document}

%\begin{table}[H]
%    \begin{center}
%        \begin{tabular}{| l | l | l | l |} 
%            \hline
%            Distorsión armónica & 2 Arm. & 3 Arm. & 4 Arm. \\
%            \hline
%            Señal A & -51 dB & -53 dB & -54 dB \\
%            \hline
%            Señal B & -76 dB & -65 dB & -44 dB \\
%            \hline
%        \end{tabular}
%    \end{center}
%    \caption{Ilustración de la edición de una tabla}
%    \label{table:ex}
%\end{table}

%\begin{figure}[H]
%    \begin{center}
%      \includegraphics[width=\linewidth, height=3cm]{../report/media/epsevg_logo.jpeg}
%    \end{center}
%    \caption{Ejemplo pie de figura}\label{fig:ex}
%\end{figure}