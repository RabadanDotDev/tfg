\newpage
\pagestyle{nonumber}

\begin{center}
    \textbf{
        RESUM
    }
    \\[\baselineskip]
\end{center}

\begin{center}
    \setlength\fboxsep{0.05\textwidth}\fbox{
        \parbox[t][\textwidth]{0.8\textwidth}{
            \quad En aquest Treball de Final de Grau (TFG) es fa el desenvolupament de PacketPincer, una eina d'anàlisis de xarxa, i es mostra el seu funcionament per mitjà d'analitzar traces de tràfic de xarxa i l'ús de les dades resultants per entrenar models amb Machine Learning.

            \quad Pel desenvolupament de l'eina s'ha fet ús de Rust i en les tasques de Machine Learning i anàlisis de dates s'ha treballat en Python. El desenvolupament s'ha realitzat en un entorn de desenvolupament dins d'un contenidor Docker. Tot el codi de l'eina, els scripts i la configuració de l'entorn s'ofereixen per permetre resultats tan reproduïbles com sigui possible.
            
            \quad L'eina és capaç de generar 72 característiques contínues sobre fluxos TCP i UDP, a més de 2 discretes indicant el protocol de transport i una identificació de cada flux composta per 7 valors. Addicionalment, l'eina permet realitzar un etiquetatge automàtic a partir d'un arxiu \acrshort{csv} on s'indica per cada fila el parell d'adreces IP, el protocol de transport utilitzat i el temps d'inici i final.
            
            \quad S'ha utilitzat l'eina desenvolupada per analitzar el tràfic de xarxa d'un conjunt de dades públic i amb les dades resultants s'ha entrenat un model fent ús de l'algoritme de Random Forest. S'ha obtingut un valor de precisió del 99.95\% i una puntuació F1 mitja del 98.66\%.
            
            \quad L'eina desenvolupada pot ser utilitzada com a component en sistemes de detecció d'intrusions i té el potencial per ser estès amb més funcionalitats en el futur.
        }
    }
\end{center}

\vspace{0.3cm}
\textbf{Paraules clau:}
\begin{center}
    \renewcommand{\arraystretch}{1.5}\begin{tabular}{|m{0.2\textwidth}|m{0.2\textwidth}|m{0.2\textwidth}|m{0.2\textwidth}|}
        \hline
            \centering\arraybackslash{Anàlisis de xarxa} & \centering\arraybackslash{Machine Learning} & \centering\arraybackslash{Ciberseguretat} & \centering\arraybackslash{Desenvolupament} \\
        \hline
        \centering\arraybackslash{Python} & \centering\arraybackslash{Rust} & \centering\arraybackslash{Docker} \\
        \cline{1-3}
    \end{tabular}
\end{center}
