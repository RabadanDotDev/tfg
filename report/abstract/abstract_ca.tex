\newpage
\pagestyle{nonumber}

\begin{center}
    \textbf{
        RESUM
    }
    \\[\baselineskip]
\end{center}

\begin{center}
    \setlength\fboxsep{0.05\textwidth}\fbox{
        \parbox[t][\textwidth]{0.8\textwidth}{
            \par{
                Amb una extensió màxima de 50 línies, i amb una llista de màxim 10 paraules clau, el resum és un text informatiu que permet decidir sobre la utilitat de llegir el document complet; ha de definir l’objectiu, els mètodes, els resultats i les conclusions presentats en el cos del document, en aquest ordre o destacant inicialment els resultats i les conclusions; ha de ser un text complet perquè sigui intel·ligible sense necessitat de referir-se a la memòria; ha de contenir la informació bàsica i el caràcter del document original. Com en tots els documents cal vetllar per la correcció d’estil, cal també emprar una nomenclatura normalitzada, i definir els termes no familiars les abreviacions i els símbols, quan apareguin per primera vegada en el resum. És la pàgina número 1 del document.
            }
        }
    }
\end{center}

\vspace{0.3cm}
\textbf{Paraules clau (màxim 10):}
\begin{center}
    \renewcommand{\arraystretch}{1.5}\begin{tabular}{|m{0.2\textwidth}|m{0.2\textwidth}|m{0.2\textwidth}|m{0.2\textwidth}|}
        \hline
            \centering\arraybackslash{?} & \centering\arraybackslash{?} & \centering\arraybackslash{?} & \centering\arraybackslash{?} \\
        \hline
        \centering\arraybackslash{?} & \centering\arraybackslash{?} & \centering\arraybackslash{?} & \centering\arraybackslash{?} \\
        \hline
        \centering\arraybackslash{?} & \centering\arraybackslash{?} \\
        \cline{1-2}
    \end{tabular}
\end{center}
