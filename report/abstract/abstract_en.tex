\newpage
\pagestyle{nonumber}

\begin{center}
    \textbf{
        ABSTRACT
    }
    \\[\baselineskip]
\end{center}

\begin{center}
    \setlength\fboxsep{0.05\textwidth}\fbox{
        \parbox[t][\textwidth]{0.8\textwidth}{
            \quad \quad This bachelor's thesis (TFG) contains the development of PacketPincer, a network analysis tool, the showcase of its usage on network traces, and the employment of the results on Machine Learning algorithms.

            \quad \quad Rust has been used for the development of the tool and Python for the data analysis process and Machine Learning. The development has been made in a containerized development environment. All the code from the tool, the scripts and the configuration of the environment are available to allow for reproducible results.
            
            \quad \quad The tool is capable of generating 72 continuous features with information about the TCP and UDP flows. In addition to the previous, it adds 2 discrete columns to indicate the used protocol and 7 values to be able to identify each detected flow. Additionally, the tool allows the user to provide a \acrshort{csv} file in order to perform automatic tagging. This file should contain, for each record, the pair of IP addresses, the used transport protocol, and the starting and ending timestamps.
            
            \quad \quad With the source datasets, the developed tool and a model based on random forests, a precision value of 99.95\% and an average F1 macro score of 98.66\% were obtained.
            
            \quad \quad The tool can be used as a component in an intrusion detection system and has the possibility to be extended with more functionalities in the future.
        }
    }
\end{center}

\vspace{0.3cm}
\textbf{Keywords:}
\begin{center}
    \renewcommand{\arraystretch}{1.5}\begin{tabular}{|m{0.2\textwidth}|m{0.2\textwidth}|m{0.2\textwidth}|m{0.2\textwidth}|}
        \hline
            \centering\arraybackslash{Network analysis} & \centering\arraybackslash{Machine Learning} & \centering\arraybackslash{Cybersecurity} & \centering\arraybackslash{Development} \\
        \hline
        \centering\arraybackslash{Python} & \centering\arraybackslash{Rust} & \centering\arraybackslash{Docker} \\
        \cline{1-3}
    \end{tabular}
\end{center}
