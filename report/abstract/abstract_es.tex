\newpage
\pagestyle{nonumber}

\begin{center}
    \textbf{
        RESUMEN
    }
    \\[\baselineskip]
\end{center}

\begin{center}
    \setlength\fboxsep{0.05\textwidth}\fbox{
        \parbox[t][\textwidth]{0.8\textwidth}{
            \quad \quad En este trabajo se hace el desarrollo de una herramienta de análisis de red y se demuestra su funcionamiento a partir de analizar trazas de tráfico de red y su uso en algoritmos de Machine Learning.
            
            \quad \quad Para el desarrollo de la herramienta se ha hecho uso de Rust y en las tareas de Machine Learning y análisis de datos se ha trabajado con Python. El desarrollo se ha realizado bajo un entorno de desarrollo en un contenedor Docker. Todo el código de la herramienta, los scripts y la configuración del entorno se ofrecen para permitir resultados lo más reproducibles posibles.
            
            \quad \quad La herramienta es capaz de generar 72 características continuas con información de flujos TCP y UDP, además de 2 discretas indicando el protocolo de transporte y una identificación de cada flujo compuesta por 7 valores. Adicionalmente, la herramienta permite realizar un etiquetado automático a partir de un fichero CSV en el cual se indica por cada fila el par de direcciones IP, el protocolo de transporte utilizado y el tiempo de inicio y final.
            
            \quad \quad A partir de los conjuntos de datos utilizados, la herramienta desarrollada y un modelo basado en bosques aleatorios, se ha obtenido un valor de precisión del 99.95\% y una puntuación F1 media del 98.66\%
            
            \quad \quad La herramienta desarrollada puede ser utilizada como componente en sistemas  de detección de intrusiones y tiene el potencial para ser extendida con más funcionalidades en el futuro.
        }
    }
\end{center}

\vspace{0.3cm}
\textbf{Palabras clave:}
\begin{center}
    \renewcommand{\arraystretch}{1.5}\begin{tabular}{|m{0.2\textwidth}|m{0.2\textwidth}|m{0.2\textwidth}|m{0.2\textwidth}|}
        \hline
            \centering\arraybackslash{Análisis de red} & \centering\arraybackslash{Machine Learning} & \centering\arraybackslash{Ciberseguridad} & \centering\arraybackslash{Desarrollo} \\
        \hline
        \centering\arraybackslash{Python} & \centering\arraybackslash{Rust} & \centering\arraybackslash{Docker} \\
        \cline{1-3}
    \end{tabular}
\end{center}
