%!TEX program = lualatex

\documentclass[11pt]{report}

% Dependencies
\usepackage{graphicx}
\usepackage{amsmath}
\usepackage{multicol}
\usepackage{amssymb}
\usepackage{tabularx}
\usepackage{xcolor}
\usepackage{fancyhdr}
\usepackage{fontspec}
\usepackage[spanish,es-tabla,catalan,english]{babel}
\usepackage{tocloft}
\usepackage[
    hyperindex=true,
    bookmarks=true,
    bookmarksnumbered=true,
    hidelinks=true,
]{hyperref}
\usepackage[all]{hypcap}

%%%%%%%%%%%%%%%%%%%%%%%%%%%%%%%%%%%%%%%%%%%%%%%%%%%%%%%%%%%%%%%%%%%%%%%%%%%%%%%
% Document style                                                              %
%%%%%%%%%%%%%%%%%%%%%%%%%%%%%%%%%%%%%%%%%%%%%%%%%%%%%%%%%%%%%%%%%%%%%%%%%%%%%%%

\usepackage[
    lmargin=3cm,
    rmargin=3cm,
    tmargin=2.5cm,
    bmargin=2.5cm,
]{geometry}
\setmainfont{Arial}
\definecolor{greyed}{RGB}{242,242,242}

\fancypagestyle{plain}{
    \fancyhf{}
    \fancyhead[L]{
        \color{gray}
        \scriptsize
        Desarrollo de una herramienta de análisis de tráfico de red y su uso en algoritmos de ML para la detección de ataques.
        \\
        Raul Rabadan Arroyo
    }
    %\renewcommand{\headrulewidth}{0pt}
    \fancyfoot[R]{\thepage}
}
\pagestyle{plain}

\fancypagestyle{nonumber}{
    \fancyhf{}
    \renewcommand{\headrulewidth}{0pt}
}

%%%%%%%%%%%%%%%%%%%%%%%%%%%%%%%%%%%%%%%%%%%%%%%%%%%%%%%%%%%%%%%%%%%%%%%%%%%%%%%
% Table of contents style                                                     %
%%%%%%%%%%%%%%%%%%%%%%%%%%%%%%%%%%%%%%%%%%%%%%%%%%%%%%%%%%%%%%%%%%%%%%%%%%%%%%%

\tocloftpagestyle{empty}

% Table of contents title
\setlength{\cftbeforetoctitleskip}{0ex}%
\setlength{\cftaftertoctitleskip}{0ex}%
\renewcommand{\cfttoctitlefont}{\hfill\textbf}%
\renewcommand{\cftaftertoctitle}{\hfill}%
\addto\captionsspanish{\renewcommand{\contentsname}{ÍNDICE}}

% Table of contents chapter
\setlength{\cftbeforechapskip}{1em}%
\setlength{\cftchapindent}{0em}%
\renewcommand{\cftchapaftersnum}{. \space}%
\renewcommand{\cftchapleader}{\hfill}%

% Table of contents section
\setlength{\cftbeforesecskip}{0.5em}%
\setlength{\cftsecindent}{0em}%
\renewcommand{\cftsecfont}{\bfseries\footnotesize}%
\renewcommand{\cftsecpagefont}{\bfseries\footnotesize}%
\renewcommand{\cftsecaftersnum}{. \space}%
\renewcommand{\cftsecleader}{\hfill}%

% Table of contents subsection
\setlength{\cftbeforesubsecskip}{0.5em}%
\setlength{\cftsubsecindent}{0em}%
\renewcommand{\cftsubsecfont}{\bfseries\footnotesize}%
\renewcommand{\cftsubsecpagefont}{\bfseries\footnotesize}%
\renewcommand{\cftsubsecaftersnum}{. \space}%
\renewcommand{\cftsubsecleader}{\hfill}%

% Table of contents subsubsection
\setlength{\cftbeforesubsubsecskip}{0.5em}%
\setlength{\cftsubsubsecindent}{0em}%
\renewcommand{\cftsubsubsecfont}{\bfseries\footnotesize}%
\renewcommand{\cftsubsubsecpagefont}{\bfseries\footnotesize}%
\renewcommand{\cftsubsubsecaftersnum}{. \space}%
\renewcommand{\cftsubsubsecleader}{\hfill}%

% List of figures title
\setlength{\cftbeforeloftitleskip}{0ex}%
\setlength{\cftafterloftitleskip}{0ex}%
\renewcommand{\cftloftitlefont}{\hfill\textbf}%
\renewcommand{\cftafterloftitle}{\hfill}%
\addto\captionsspanish{\renewcommand{\listfigurename}{ÍNDICE DE FIGURAS}}

% List of figures point
\renewcommand{\cftfigleader}{\hfill}%
\setlength{\cftfignumwidth}{5em}
\renewcommand{\cftfigpresnum}{Figura~}
\renewcommand{\cftfigaftersnum}{.}

% List of tables title
\setlength{\cftbeforelottitleskip}{0ex}%
\setlength{\cftafterlottitleskip}{0ex}%
\renewcommand{\cftlottitlefont}{\hfill\textbf}%
\renewcommand{\cftafterlottitle}{\hfill}%
\addto\captionsspanish{\renewcommand{\listtablename}{ÍNDICE DE TABLAS}}

% List of tables point
\renewcommand{\cfttableader}{\hfill}%
\setlength{\cfttabnumwidth}{5em}
\renewcommand{\cfttabpresnum}{Tabla~}
\renewcommand{\cfttabaftersnum}{.}

%%%%%%%%%%%%%%%%%%%%%%%%%%%%%%%%%%%%%%%%%%%%%%%%%%%%%%%%%%%%%%%%%%%%%%%%%%%%%%%
% Glossary                                                                    %
%%%%%%%%%%%%%%%%%%%%%%%%%%%%%%%%%%%%%%%%%%%%%%%%%%%%%%%%%%%%%%%%%%%%%%%%%%%%%%%

% Import
\usepackage[acronym,automake,section=subsubsection]{glossaries}

% Generation
\makeglossaries

% Contents
\newglossaryentry{matrix}% the label
{name={matrix},% the term
 description={a rectangular table of elements},% brief description
 plural={matrices}% the plural
}

\newacronym{svm}{SVM}{support vector machine}

% Style

\addto\captionsspanish{\renewcommand{\glossaryname}{TÉRMINOS}}
\addto\captionsspanish{\renewcommand{\acronymname}{ACRÓNIMOS}}

%%%%%%%%%%%%%%%%%%%%%%%%%%%%%%%%%%%%%%%%%%%%%%%%%%%%%%%%%%%%%%%%%%%%%%%%%%%%%%%
% Document                                                                    %
%%%%%%%%%%%%%%%%%%%%%%%%%%%%%%%%%%%%%%%%%%%%%%%%%%%%%%%%%%%%%%%%%%%%%%%%%%%%%%%

\begin{document}

% Front pages
\selectlanguage{spanish}
\begin{titlepage}

    \centering{\includegraphics{media/epsevg_logo.jpeg}}

    \vspace{1.5cm}
    \centering{\Huge\bfseries TREBALL FINAL DE GRAU}

    \vfill
    \textbf{TÍTOL} \\ \textbf{Desarrollo de un Sistema de Detección de Intrusiones en Redes IoT a través de algoritmos de Aprendizaje por Refuerzo.} 
    \\[\baselineskip]
    \textbf{AUTOR:} \\ \textbf{Rabadan Arroyo, Raul} 
    \\[\baselineskip]
    \textbf{DATA PRESENTACIÓ:} \\ \textbf{Juny, 2024}
    \vfill

\end{titlepage}

\begin{titlepage}
    \centering{
        \setlength\fboxsep{0.5cm}\fbox{
            \parbox{13.5cm}{
                \par{
                    \bfseries COGNOMS: RABADAN ARROYO \quad\quad\quad\quad\quad\quad\quad NOM: RAUL \\[\baselineskip]
                    \bfseries TITULACIÓ: Grau en Enginyeria Informàtica \\[\baselineskip]
                    \bfseries PLA: 2018 \\[\baselineskip]
                    \bfseries DIRECTOR: ? \\[\baselineskip]
                    \bfseries DEPARTAMENT: ?
                }
            }
        }
    }
    
    \vfill

    \centering{
        \setlength\fboxsep{0cm}\colorbox{greyed}{
            \setlength\fboxsep{0.5cm}\fbox{
                \parbox[t]{7.5cm}{
                    \centering{
                        \textbf{
                            QUALIFICACIO DEL TFG
                        }
                    }
                    \vspace {
                        2cm
                    }
                }
            }
        }
    }

    \vfill

    \centering{
        \setlength\fboxsep{0cm}\colorbox{greyed}{
            \setlength\fboxsep{0.5cm}\fbox{
                \parbox{13.5cm}{
                    \centering{
                        \underline{
                            \textbf{
                                TRIBUNAL
                            }
                        }
                    } 
                    \\[\baselineskip]
                    \centering{
                        \hfill
                        \textbf{PRESIDENT}
                        \hfill
                        \textbf{SECRETARI}
                        \hfill
                        \textbf{VOCAL}
                        \hfill
                    }
                    \vspace {
                        2.5cm
                    }
                }
            }
        }
    }

    \vfill
    
    \centering{
        \setlength\fboxsep{0cm}\colorbox{greyed}{
            \setlength\fboxsep{0.5cm}\fbox{
                \parbox{13.5cm}{
                    \vspace {
                        0.5cm
                    }
                    \textbf{
                        DATA DE LECTURA:
                    }
                    \vspace {
                        0.5cm
                    }
                }
            }
        }
    }

    \vfill

    \centering{
        \textbf{
            Aquest Projecte té en compte aspectes mediambientals: \square ~ Sí ~ \blacksquare ~ No
        }
    }
    
\end{titlepage}

% Abatracts
\selectlanguage{spanish}
\newpage
\pagestyle{nonumber}

\begin{center}
    \textbf{
        RESUMEN
    }
    \\[\baselineskip]
\end{center}

\begin{center}
    \setlength\fboxsep{0.05\textwidth}\fbox{
        \parbox[t][\textwidth]{0.8\textwidth}{
            \quad En este Trabajo de Final de Grado (TFG) se hace el desarrollo de PacketPincer, una herramienta de análisis de red, y se demuestra su funcionamiento a partir de analizar trazas de tráfico de red y el uso de los datos resultantes para entrenar modelos con Machine Learning.
            
            \quad Para el desarrollo de la herramienta se ha hecho uso de Rust y en las tareas de Machine Learning y análisis de datos se ha trabajado con Python. El desarrollo se ha realizado bajo un entorno de desarrollo en Docker. Todo el código de la herramienta, los scripts y la configuración del entorno se ofrecen para permitir resultados lo más reproducibles posible.
            
            \quad La herramienta es capaz de generar 72 características continuas con información de flujos TCP y UDP, además de 2 discretas indicando el protocolo de transporte y una identificación de cada flujo compuesta por 7 valores. Adicionalmente, la herramienta permite realizar un etiquetado automático a partir de un fichero \acrshort{csv} en el cual se indica por cada fila el par de direcciones IP, el protocolo de transporte y el tiempo de inicio y final.
            
            \quad Se ha utilizado la herramienta desarrollada para analizar el tráfico de red de un conjunto datos públicos y con los datos resultantes se ha entrenado un modelo utilizando el algoritmo Random Forest, obteniéndose un valor de precisión del 99.95\% y una puntuación F1 media del 98.66\%
            
            \quad La herramienta desarrollada puede ser utilizada como componente en sistemas  de detección de intrusiones y tiene el potencial para ser extendida con más funcionalidades en el futuro.
        }
    }
\end{center}

\vspace{0.3cm}
\textbf{Palabras clave:}
\begin{center}
    \renewcommand{\arraystretch}{1.5}\begin{tabular}{|m{0.2\textwidth}|m{0.2\textwidth}|m{0.2\textwidth}|m{0.2\textwidth}|}
        \hline
            \centering\arraybackslash{Análisis de red} & \centering\arraybackslash{Machine Learning} & \centering\arraybackslash{Ciberseguridad} & \centering\arraybackslash{Desarrollo} \\
        \hline
        \centering\arraybackslash{Python} & \centering\arraybackslash{Rust} & \centering\arraybackslash{Docker} \\
        \cline{1-3}
    \end{tabular}
\end{center}

\selectlanguage{catalan}
\newpage
\pagestyle{nonumber}

\begin{center}
    \textbf{
        RESUM
    }
    \\[\baselineskip]
\end{center}

\begin{center}
    \setlength\fboxsep{0.05\textwidth}\fbox{
        \parbox[t][\textwidth]{0.8\textwidth}{
            \par{
                Amb una extensió màxima de 50 línies, i amb una llista de màxim 10 paraules clau, el resum és un text informatiu que permet decidir sobre la utilitat de llegir el document complet; ha de definir l’objectiu, els mètodes, els resultats i les conclusions presentats en el cos del document, en aquest ordre o destacant inicialment els resultats i les conclusions; ha de ser un text complet perquè sigui intel·ligible sense necessitat de referir-se a la memòria; ha de contenir la informació bàsica i el caràcter del document original. Com en tots els documents cal vetllar per la correcció d’estil, cal també emprar una nomenclatura normalitzada, i definir els termes no familiars les abreviacions i els símbols, quan apareguin per primera vegada en el resum. És la pàgina número 1 del document.
            }
        }
    }
\end{center}

\vspace{0.3cm}
\textbf{Paraules clau (màxim 10):}
\begin{center}
    \renewcommand{\arraystretch}{1.5}\begin{tabular}{|m{0.2\textwidth}|m{0.2\textwidth}|m{0.2\textwidth}|m{0.2\textwidth}|}
        \hline
            \centering\arraybackslash{?} & \centering\arraybackslash{?} & \centering\arraybackslash{?} & \centering\arraybackslash{?} \\
        \hline
        \centering\arraybackslash{?} & \centering\arraybackslash{?} & \centering\arraybackslash{?} & \centering\arraybackslash{?} \\
        \hline
        \centering\arraybackslash{?} & \centering\arraybackslash{?} \\
        \cline{1-2}
    \end{tabular}
\end{center}

\selectlanguage{english}
\newpage
\pagestyle{nonumber}

\begin{center}
    \textbf{
        ABSTRACT
    }
    \\[\baselineskip]
\end{center}

\begin{center}
    \setlength\fboxsep{0.05\textwidth}\fbox{
        \parbox[t][\textwidth]{0.8\textwidth}{
            \par{
                ?
            }
        }
    }
\end{center}

\par {
    \textbf{
       Keywords (10 maximum):
    }
}

\vspace{0.3cm}
\begin{center}
    \renewcommand{\arraystretch}{1.5}\begin{tabular}{|m{0.2\textwidth}|m{0.2\textwidth}|m{0.2\textwidth}|m{0.2\textwidth}|}
        \hline
            \centering\arraybackslash{?} & \centering\arraybackslash{?} & \centering\arraybackslash{?} & \centering\arraybackslash{?} \\
        \hline
        \centering\arraybackslash{?} & \centering\arraybackslash{?} & \centering\arraybackslash{?} & \centering\arraybackslash{?} \\
        \hline
        \centering\arraybackslash{?} & \centering\arraybackslash{?} \\
        \cline{1-2}
    \end{tabular}
\end{center}


% Indices
\selectlanguage{spanish}
\selectlanguage{spanish}
\tableofcontents
\newpage
\listoffigures
\newpage
\setglossarystyle{listgroup}
\hfill\textbf{GLOSARIO}\hfill
\printglossaries


% Contents
\newpage
\pagestyle{plain}

\phantomsection\addcontentsline{toc}{chapter}{INTRODUCCIÓN}
\chapter*{INTRODUCCIÓN}

La ciberseguridad es una de las fronteras del conocimiento que ha tomado más relevancia estos últimos años. Para salvaguardar la disponibilidad, integridad y confidencialidad de la información tanto almacenada como en tránsito, se aplican numerosas y diversas técnicas en conjunto. Desde medidas criptográficas para proteger la información hasta el análisis del tráfico de red para detectar comportamientos maliciosos y poder bloquearlos. El presente trabajo se focalizará principalmente en el último punto.

\phantomsection\addcontentsline{toc}{section}{MOTIVACIÓN}
\section*{MOTIVACIÓN}

La motivación del ámbito de este trabajo surge de una beca de colaboración con el grupo de investigación CRAAX y mi interés por la aplicabilidad de los sistemas de \gls{ml} en entornos limitados o con requerimientos de actuación en tiempo real, especialmente en el ámbito de ciberseguridad en las redes. En muchos casos, los tráficos de red maliciosos son identificados cuando estos ya han ocurrido o están generando problemas activamente para el resto de usuarios de la red. La gran utilidad que supondría la detección de ataques en su inicio o incluso antes de que ocurriesen es una meta en la que me gustaría colaborar. Llegar a este ideal es altamente difícil. Sin embargo, tenerlo como horizonte para dirigir el camino y acercarse lo máximo posible a este, ofrece la capacidad de mejorar la mitigación contra posibles adversarios.

Durante la colaboración, he utilizado herramientas de extracción de características para la caracterización de flujos de red. Sin embargo, estas presentaban ciertas limitaciones e inconvenientes, los cuales me han impulsado a querer desarrollar una alternativa.

\phantomsection\addcontentsline{toc}{section}{OBJETIVOS}
\section*{OBJETIVOS}

El objetivo principal del trabajo consiste en diseñar, programar y demostrar la utilidad de una herramienta de análisis de red basada en la extracción de características de esta para detectar comportamientos maliciosos. La herramienta requerirá de ser robusta y eficiente, además de fácil de utilizar, extender y modificar. La robusteza y eficiencia son necesarias, ya que por su naturaleza tendrá que tratar datos en tiempo real en entornos limitados. La facilidad de uso, extensión y modificación es importante, ya que en el desarrollo de aplicaciones, y especialmente en el ámbito de la ciberseguridad, el entorno y los requerimientos están en constante variación.

La utilidad de la herramienta se evaluará según su capacidad de adaptación a diferentes entornos de ejecución y diferentes tipos de tráfico de red. Esto es necesario, ya que, dependiendo de su rendimiento, podrá ser integrada en sistemas más generales y complejos.

\phantomsection\addcontentsline{toc}{section}{PASOS PARA EL DESAROLLO}
\section*{PASOS PARA EL DESAROLLO}

\begin{itemize}
  \item Anàlisis de las herramientas de extracción de características
  \begin{itemize}
    \item CICFlowmeter-V4.0 (java)
    \item CICFlowmeter-V4.0 (python)
    \item nProbe
    \item Identificación de las posibles mejoras
  \end{itemize}
  \item Anàlisis de datasets a utilizar
  \begin{itemize}
    \item CICFlowmeter-V4.0 (java)
    \item CICFlowmeter-V4.0 (python)
    \item nProbe
    \item Identificación de las posibles mejoras
  \end{itemize}
  \item Desarrollo de la herramienta
  \begin{itemize}
    \item Lectura del formato pcap
    \item Lectura de paquetes en vivo
    \item Extracción de campos de paquetes
    \item Separación de paquetes en flujos
    \item Computo de características de flujos
    \item Exportación de las características de los flujos
    \item Etiquetado de los flujos
    \item Uso de la herramienta como libreria de python ?
  \end{itemize}
  \item Demostración del funcionamiento de la herramienta
  \begin{itemize}
    \item Generación de datos de entrenamiento con flujos terminados
    \item Entrenamiento de algoritmos de ML con flujos terminados
    \item Generación de datos de entrenamiento con flujos parciales ?
    \item Entrenamiento de algoritmos de ML con flujos parciales ?
  \end{itemize}
\end{itemize}

\phantomsection\addcontentsline{toc}{section}{METODOLOGÍA}
\section*{METODOLOGÍA}

(por explicar)

\newpage
\pagestyle{plain}

\chapter{CHAPTER TEST}

TODO
\section{SECTION TEST}

TODO

\subsection{SUBSECTION TEST}

TODO

\newpage
\pagestyle{plain}

\chapter{CONCLUSIONES}

Como recapitulación del trabajo, primero haremos un recuento aproximado de los posibles costes que se hubiesen incurrido en caso de haber sido un proyecto empresarial. Después de esto, se hará un análisis de los resultados generales obtenidos y continuaremos con indicaciones sobre posible trabajo futuro. Finalmente, concluiremos con los agradecimientos.

\section{Costes}

\begin{table}[H]
  \centering
  \begin{tabular}{|l | r |}
      \hline
      \rowcolor{lightgray} \textbf{Tarea}                & \textbf{Horas}       \\  
      \hline
      \rowcolor{lightgray} Gestión del proyecto     &  45                  \\
      Reuniones de seguimiento                      &  35                  \\
      Gestión de las tareas                         &   5                  \\
      Planificación temporal                        &   5                  \\  
      \hline
      \rowcolor{lightgray} Investigación            & 125                  \\
      Herramientas de extracción de características  &  75                  \\
      Conjuntos de datos                            &  30                  \\
      Formatos y protocolos de red                  &  10                  \\
      Otro software utilizado                       &  10                  \\
      \hline
      \rowcolor{lightgray} Desarrollo                & 220                  \\
      Programación de la herramienta                & 110                  \\
      Extensión de la librería de código abierto    &  30                  \\
      Modelos de ML                                 &  80                  \\
      \hline
      \rowcolor{lightgray} Documentación            & 145                  \\
      Redacción de la memoria                       & 110                  \\
      Artículo                                      &  20                  \\
      Presentación                                  &  15                  \\
      \hline
      \rowcolor{lightgray} Total                    & 535                  \\
      \hline
  \end{tabular}
  \caption{Horas aproximadas dedicadas a diferentes partes del proyecto}
  \label{table:horasdedicadas}
\end{table}

En la Tabla \ref{table:horasdedicadas} se puede observar un desglose aproximado de las horas utilizadas para el desarrollo del proyecto. Debido a que se estaban cursando asignaturas y colaborando con el grupo de Investigación CRAAX al mismo tiempo, es posible que haya habido interferencias y el número real sea inferior o superior. 

Respecto al coste por horas, podemos ver en la Tabla \ref{table:salarios} un posible coste de la realización del trabajo, tomando como referencia diferentes salarios medios según la tarea realizada. El total asciende a 13 447.47, aunque los tiempos requeridos y los salarios podrían haber sido distintos en un entrono empresarial.

\begin{table}[H]
  \centering
  \begin{tabular}{|l | c c c |}
      \hline
      \rowcolor{lightgray} \textbf{Rol} & \textbf{Salario medio por hora}            & \textbf{Horas totales} & \textbf{Coste total} \\ \hline
      Ingeniero de proyectos            & 14.10 € \cite{salarioingenierodeproyectos} &  45                    &    634.5  €          \\
      Programador                       & 14.36 € \cite{salarioprogramador}          & 235                    &  9 111.42 €          \\
      Científico de datos               & 20.64 € \cite{salariodatasci}              & 110                    &  2 270.4  €          \\
      Redactor técnico                  &  9.87 € \cite{salarioredactor}             & 145                    &  1 431.15 €          \\ \hline
      \rowcolor{lightgray}              &                                           &                        & \textbf{13 447.47 €}  \\
      \hline
  \end{tabular}
  \caption{Cálculo coste horas por tarea realizada}
  \label{table:salarios}
\end{table}

No se han incurrido en costes de licencias, ya que todo el software utilizado ha sido gratuito o de código libre. Para el caso del ordenador de sobremesa, donde se han ejecutado los algoritmos y se ha hecho la mayor parte del desarrollo, está valorado en unos 2 500 €.

Si hacemos la suma total, el coste total sería de unos 16 000 €. Cabe notar que esta suma se ha realizado sin tener en cuenta costes indirectos como la factura eléctrica.

\section{Análisis resultados}

por hacer

\section{Trabajo futuro}

por hacer

\section{Agradecimientos}

por hacer

\newpage
\pagestyle{plain}

\phantomsection\addcontentsline{toc}{chapter}{AGRADECIMIENTOS}
\chapter*{AGRADECIMIENTOS}

TODO

\newpage
\pagestyle{plain}

\phantomsection\addcontentsline{toc}{chapter}{BIBLIOGRAFÍA}
\chapter*{BIBLIOGRAFÍA}

TODO

\newpage
\pagestyle{plain}

\pdfbookmark[-1]{ANEXOS}{ANEXOS}

\cftaddtitleline{toc}{chapter}{ANEXO 1}{}
\pdfbookmark[0]{ANEXO 1}{ANEXO 1}
\chapter*{ANEXO 1}
\setcounter{page}{0}

TODO

\cftaddtitleline{toc}{chapter}{ANEXO 2}{}
\pdfbookmark[0]{ANEXO 2}{ANEXO 2}
\chapter*{ANEXO 2}
\setcounter{page}{0}

TODO


\end{document}
