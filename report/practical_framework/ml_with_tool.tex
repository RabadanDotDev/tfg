\section{Uso de la herramienta}

En esta sección trataremos el proceso de hacer uso de la herramienta para el entrenamiento de modelos. Primero extraeremos las etiquetas de cada dataset para posteriormente ejecutar nuestra herramienta sobre los datos en crudo. Una vez tengamos las características extraídas, haremos una fase de preprocesamiento de los datos donde analizaremos las diferentes propiedades de este, escalando y normalizando los datos donde posible. Con los datos limpiados, realizaremos una selección de características para mantener solo las que se consideren más relevantes. Una vez hecho esto, detallaremos la tarea de \gls{ml} que se realizará y como evaluaremos los diferentes modelos. A continuación, entrenaremos diferentes modelos y valoraremos su efectividad. Finalmente, haremos una comparación de los modelos y decidiremos cuál es el que muestra mejor rendimiento.

\subsection{Extracción etiquetas de los datasets}

Por hacer

\subsection{Ejecución con etiquetado}

Por hacer

\subsection{Preprocessamiento datos generados}

por hacer

\subsection{Definición y evaluación de la tarea a realizar}

por hacer

\subsection{Entrenamiento de modelos}

por hacer

\subsection{Comparación de resultados}

por hacer

