\newpage
\pagestyle{plain}

\phantomsection\addcontentsline{toc}{chapter}{INTRODUCCIÓN}
\chapter*{INTRODUCCIÓN}

La ciberseguridad es una de las fronteras del conocimiento que ha tomado más relevancia estos últimos años. Para salvaguardar la disponibilidad, integridad y confidencialidad de la información tanto almacenada como en tránsito, se aplican numerosas y diversas técnicas en conjunto. Desde medidas criptográficas para proteger la información hasta el análisis del tráfico de red para detectar comportamientos maliciosos y poder bloquearlos. El presente trabajo se focalizará en el último punto.

\phantomsection\addcontentsline{toc}{section}{MOTIVACIÓN}
\section*{MOTIVACIÓN}

La motivación de este trabajo radica en mi interés por la aplicabilidad de los sistemas de \gls{ml} en entornos limitados o con requerimientos de actuación en tiempo real, especialmente en el ámbito de ciberseguridad en las redes. En muchos casos, los tráficos de red maliciosos son identificados cuando estos ya han ocurrido o están generando problemas activamente para el resto de usuarios de la red La hazaña intelectual que supondría la detección de ataques en su inicio o incluso antes de que ocurriesen es una meta en la que me gustaría colaborar. Llegar a este ideal es seguramente improbable. Sin embargo, tenerlo como horizonte para dirigir el camino y acercarse lo máximo posible a este, ofrece la capacidad de mejorar la mitigación contra posibles adversarios.

Existen diversas herramientas similares a la ideada para el trabajo, pero debido a su dificultad de uso, su naturaleza propietaria o su desalineación al uso específico que se le quiere dar, he querido desarrollar una alternativa.

\phantomsection\addcontentsline{toc}{section}{OBJETIVOS}
\section*{OBJETIVOS}

El objetivo principal del trabajo consiste en diseñar, programar y demostrar la utilidad de una herramienta de análisis de red basada en la extracción de características de esta para detectar comportamientos maliciosos. La herramienta requerirá de ser robusta y eficiente, además de fácil de utilizar, extender y modificar. La robusteza y eficiencia son necesarias, ya que por su naturaleza tendrá que tratar datos en tiempo real en entornos limitados. La facilidad de uso, extensión y modificación es importante, ya que en el desarrollo de aplicaciones, y especialmente en el ámbito de la ciberseguridad, el entorno y los requerimientos están en constante variación.

La utilidad de la herramienta se evaluará según su capacidad de adaptación a diferentes entornos de ejecución y diferentes tipos de tráfico de red. Esto es necesario, ya que, dependiendo de su rendimiento, podrá ser integrada en sistemas más generales y complejos.

\phantomsection\addcontentsline{toc}{section}{METODOLOGÍA}
\section*{METODOLOGÍA}

(extreme programming, definir tests y metas etc)