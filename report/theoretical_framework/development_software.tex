\section{Software para el desarrollo}

En esta sección resumiremos los diferentes elementos software que serán utilizados para el desarrollo del trabajo y no hayan sido mencionados en otros puntos. Concretamente, se indicarán los lenguajes de programación utilizados y los programas excepto Tshark, ya que se explicó en la sección \ref{tshark}, además de la razón de su uso.

\subsection{Lenguajes de programación}

\subsubsection{Rust}

Rust es un lenguaje de programación compilado de propósito general y multiparadigma \cite{blandy2017programming} \cite{klabnik2018rust}. Este trata de garantizar que todas las referencias a memoria sean válidas e impedir condiciones de carrera sin tener necesidad de un 'recolector de basura' como Java o Python. Esto lo hace en el momento de compilación, a través de comprobar que los usos de memoria sean correctos. 

Debido a que garantizar con exactitud si un programa hace uso de la memoria de forma correcta es un reto, el compilador toma una posición más conservadora. En caso de que el programador tenga más información y sepa que un uso de memoria que el compilador rechaza es correcto, existen métodos para hacer que el compilador lo acepte. Ejemplos son el uso de \texttt{unwrap()} para acceder a valores opcionales, el cual, si no contiene el valor esperado, interrumpe el programa en vez de corromper la memoria, o bloques \texttt{unsafe}, donde el programador ha de manualmente comprobar que no se está accediendo a memoria inválida.

Se ha escogido este lenguaje para el desarrollo de la herramienta debido a la capacidad de escribir código de bajo nivel, su alto rendimiento, su énfasis en el código correcto, la calidad de las herramientas asociadas y el gran número de librerías disponibles. 

\subsubsection{Bash}

Bash, también conocido como GNU Bash, es un programa Shell y un lenguaje de comandos asociado que formaba originalmente parte del sistema operativo GNU \cite{gnubashweb} \cite{gnubashmanual}. Se puede ejecutar de forma interactiva, donde escribimos los comandos directamente en el terminal, o de manera no interactiva, donde le pasamos un archivo con la lista de comandos a ejecutados. Los comandos pueden ser palabras intrínsecas del lenguaje (\texttt{if}, \texttt{do}, entre otros) o programas para ser ejecutado. Adicionalmente, tiene soporte para bucles, guardar y expandir variables, funciones, listas, ente otros.

Se ha escogido Bash porque es la Shell utilizada por defecto en muchas distribuciones Linux. Con esta, automatizaremos tareas que requieran tratar archivos con herramientas existentes, como por ejemplo \texttt{tshark}.

\subsubsection{Python}

Python es un lenguaje de programación interpretado de propósito general y multiparadigma \cite{aboutpython} \cite{davepython}. Trata de focalizarse en la legibilidad del código y su facilidad de uso en la medida de lo posible. Esto lo hace a través de una librería estándar extensa, una gran cantidad de librerías creadas por la comunidad, tipos dinámicos y un 'recolector de basura' que permite a los desarrolladores no tener que gestionar memoria de forma manual.

Se ha escogido este lenguaje para las tareas que consistan en visualizar y tratar datos, además de las tareas de \gls{ml}, ya que es uno de los lenguajes frecuentemente utilizados para esto. Adicionalmente, las librerías disponibles y la documentación asociada facilitarán su uso.

\subsubsection{LaTeX}

LaTeX es un sistema de composición de textos como el utilizado comúnmente para la creación de documentos científicos y técnicos \cite{latexweb}. LaTeX no es un procesador de textos como Word o Google Docs donde se puede ver el 'resultado' del documento mientras lo editas, sino que se asemeja más a HTML o Markdown, el cual ha de ser 'renderizado' para poder ver el documento con el formato aplicado. LaTeX se focaliza en que los autores se concentren en desarrollar el contenido primero y luego en definir el estilo \cite{latexabout}.

Se utilizará LaTeX para la escritura de la memoria y otros documentos del proyecto debido a que se puede mantener los archivos fuente en control de versiones, además de que permite adherirse al formato esperado de la memoria.

\subsection{Herramientas}

\subsubsection{Git}

Git es un sistema de versión de control distribuido, que mantiene versiones de archivos y su evolución en el tiempo \cite{chacon2014pro}. Se considera un sistema distribuido, ya que no requiere de un servidor central que gestione el control de versiones, sino que se hace en cada nodo y a continuación se sincroniza el estado. Tiene una gran flexibilidad, permitiendo, entre muchas cosas, tener versiones alternativas del estado del código en 'ramas', revertir cambios, encontrar cuando se modificó una línea y su historia. Adicionalmente, todas las actualizaciones tienen comprobaciones de integridad y se pueden opcionalmente firmar criptográficamente.

Se utilizará este programa para realizar el control de versiones de todo el proyecto. Desde la redacción de la memoria, el desarrollo de la herramienta y los scripts en Bash o Python asociados. Se ha escogido este debido a su popularidad y por el hecho de que servicios como GitHub permiten mantener una copia de todo el historial en la nube.

\subsubsection{Docker}

Docker es la de facto plataforma para gestionar contenedores, la cual facilita el desarrollo y el despliegue de aplicaciones de forma fácil y consistente \cite{dockerruntime}. El concepto de contenedores no es específico de Docker, sino algo más general. Según su web, un contenedor es una unidad estándar de software que empaqueta código y todas sus dependencias para que una aplicación determinada se inicie rápidamente y fiablemente en diferentes entornos \cite{dockerwhatsacontainer}. De esta manera, el entorno donde se ejecuta la aplicación es consistente y se eliminan los problemas derivados de tener instaladas librerías incompatibles. Adicionalmente, las aplicaciones se encuentran aisladas, ofreciendo una capa adicional de seguridad.

Debido a que es posible que se utilicen diferentes máquinas durante el desarrollo del trabajo, y para asegurar que el entorno es reproducible, se hará uso del sistema de contenedores ofrecido por Docker. Se utilizará este, ya que es uno de los más populares y de la experiencia utilizando este en proyectos anteriores.

\subsubsection{Visual Studio Code}

Visual Studio Code es uno de los editores de código más populares, según la encuesta a desarrolladores realizada por StackOverflow \cite{2023devsurvey}. Tiene soporte para un número elevado de lenguajes de programación, además de incorporar un sistema de extensiones que permite extender en gran medida su funcionalidad \cite{vscodeweb}.

Se ha escogido este editor para la realización del proyecto por su gran versatilidad y por ofrecer extensiones que dan soporte para todos los lenguajes que se utilizarán y sus compiladores e interpretadores asociados.

\color{blue}  %TODO remove this when revised
\subsubsection{PlantUML}

PlantUML es una herramienta de código abierto, con un lenguaje asociado, que facilita la creación de diagramas de diferentes tipos \cite{plantumlweb}. Soporta los diagramas usuales de UML de secuencia, clases, entre otros, además de diagramas de Gantt, expresiones regulares, datos en formato \acrshort{json} y más.

Se ha escogido esta herramienta para la realización del proyecto por su capacidad de generar los diagramas que se necesitan, además del hecho que se puede ejecutar en local.
\color{black}  %TODO remove this when revised
