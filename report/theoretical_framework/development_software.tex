\section{Software para el desarrollo}

En esta sección resumiremos los diferentes elementos software que serán utilizados para el desarrollo del trabajo. Concretamente, se indicarán los lenguajes de programación utilizados y los programas, además de la razón de su uso.

\subsection{Lenguajes de programación}

\subsubsection{Rust}

Rust es un lenguaje de programación compilado de propósito general y multiparadigma \cite{blandy2017programming} \cite{klabnik2018rust}. Este trata de garantizar que todas las referencias a memoria sean válidas e impedir condiciones de carrera sin tener necesidad de un 'recolector de basura' como Java o Python. Esto lo hace a través de, en el momento de compilación, comprobar que los usos de memoria sean correctos. 

Debido a que garantizar con exactitud si un programa hace uso de la memoria de forma correcta es un reto, el compilador toma una posición más conservadora. En caso de que el programador tenga más información y sepa que un uso de memoria que el compilador rechaza es correcto, existen métodos para hacer que el compilador lo acepte. Ejemplos son el uso de \texttt{unwrap()} para acceder a valores opcionales, el cual, si no contiene el valor esperado, interrumpe el programa en vez de corromper la memoria, o bloques \texttt{unsafe}, donde el programador ha de manualmente comprobar que no se está accediendo a memoria inválida.

Se ha escogido este lenguaje para el desarrollo de la herramienta debido a la capacidad de escribir código de bajo nivel, su alto rendimiento, su énfasis en el código correcto, la calidad de las herramientas asociadas y el gran número de librerías disponibles. 

\subsubsection{Bash}

\color{blue} %TODO remove this when revised

Bash, también conocido como GNU Bash, es un programa Shell y un lenguaje de comandos asociado que formaba originalmente parte del sistema operativo GNU \cite{gnubashweb} \cite{gnubashmanual}. Se puede ejecutar de forma interactiva, donde escribimos los comandos directamente en el terminal, o de manera no interactiva, donde le pasamos un archivo con la lista de comandos a ejecutar. Los comandos pueden ser palabras intrínsecas del lenguaje (\texttt{if}, \texttt{do}, entre otros) o programas para ser ejecutado. Adicionalmente, tiene soporte para bucles, guardar y expandir variables, funciones, listas, ente otros.

Se ha escogido Bash porque es la Shell utilizada por defecto en muchas distribuciones Linux. Con esta, automatizaremos tareas que requieran tratar archivos con herramientas existentes, como por ejemplo \texttt{tshark}.

\subsubsection{Python}

Python es un lenguaje de programación interpretado de propósito general y multiparadigma \cite{aboutpython} \cite{davepython}. Trata de focalizarse en la legibilidad del código y su facilidad de uso en la medida de lo posible. Esto lo hace a través de una librería estándar extensa, una gran cantidad de librerías creadas por la comunidad, tipos dinámicos y un 'recolector de basura' que permite a los desarrolladores no tener que gestionar memoria de forma manual.

Se ha escogido este lenguaje para las tareas que consistan en visualizar y tratar datos, además de las tareas de \gls{ml}, ya que es uno de los lenguajes frecuentemente utilizados para esto. Adicionalmente, las librerías disponibles y la documentación asociada facilitarán su uso.

\subsection{Programas}

\subsubsection{Git}

por hacer

\subsubsection{Docker}

por hacer

\subsubsection{Visual Studio Code}

por hacer
